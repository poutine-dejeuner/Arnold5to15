% compile with XeLaTeX or LuaLaTeX
\input{preamble}
\setdefaultlanguage{french}
% \setdefaultlanguage[variant=canada]{french}

\title{Problèmes pour enfants de 5 à 15 ans}

\author{V.\,I.~Arnold
\vspace*{2cm}\\
\includegraphics[width=\linewidth]{resources/photo-arnold_small}
}
\date{}

\begin{document}
\maketitle
\thispagestyle{empty}
\cleardoublepage
\setcounter{page}{1}
\begin{abstract}
	Cette brochure propose 77 exercices pour développer une culture de la pensée,
	sélectionnés ou élaborés par l'auteur. La plupart d'entre eux ne requièrent aucune
	connaissance particulière au­delà de la culture générale. Cependant, la résolution
	de certains exercices peut s'avérer difficile, même pour des enseignants. Cet
	ouvrage s'adresse
	aux écoliers, étudiants, enseignants, parents – à tous ceux qui considèrent la
	culture de la pensée comme un élément essentiel du développement personnel.
	nous.
\end{abstract}
\clearpage

\section*{Préface}
J'ai écrit ces tâches à Paris au printemps 2004, lorsque des Parisiens russes m'ont demandé d'aider leurs jeunes enfants à acquérir une culture de pensée traditionnelle russe, mais bien supérieure à toutes les coutumes occidentales. Je suis profondément convaincu que cette culture est mieux nourrie par une réflexion indépendante et précoce sur des questions simples, mais non faciles, telles que celles ci-dessous (je recommande particulièrement les problèmes 1, 3, 13). Ma longue expérience m'a montré que les élèves médiocres les résolvent souvent mieux que les excellents, car dans leur « Kamtchatka », ils doivent réfléchir davantage pour survivre que pour « gouverner Séville et Grenade », comme le disait lui­même Figaro, tandis que les excellents élèves ne comprennent pas « ce qui doit être multiplié par quoi » dans ces problèmes. J'ai même remarqué que les enfants de cinq ans résolvent ces problèmes mieux que les écoliers gâtés par les exercices, pour qui ils sont plus faciles que les étudiants qui ont bachoté à l'université, mais qui surpassent néanmoins leurs professeurs (les lauréats du prix Nobel et du prix Fields sont les plus maladroits a résoudre ces problèmes simples).


\clearpage
\section*{Les problèmes}

\begin{problem}{1.}
Il manquait sept kopecks à Macha pour acheter un livre, et il manquait un kopeck à Micha. Ils ont mis leur argent en commun pour l'acheter pour eux deux, mais il n'y avait toujours pas assez d'argent. Combien a coûté le livre ?
\end{problem}

\begin{problem}{2.}
Une bouteille avec un bouchon coûte 10 kopecks, et la bouteille est 9 kopecks plus chère que le bouchon. Combien coûte une bouteille sans bouchon ?
\end{problem}

\begin{problem}{3.}
Une brique pèse une livre et la demie du poids d'une brique. Combien de livres pèse une brique ?
\end{problem}

\begin{problem}{4.}
Une cuillerée de vin a été versée d'un tonneau dans une verre (non plein) de thé. Puis la même cuillerée de mélange (hétérogène) du verre a été reversée dans le tonneau. Maintenant, le tonneau et le verre contiennent tous deux un certain volume de liquide étranger (vin dans le verre, thé dans le tonneau). Où le volume de liquide étranger est-il le plus grand : dans le verre ou dans le tonneau ?
\end{problem}

\begin{problem}{5.}
Deux vieilles femmes sont parties de $A$ à $B$ et de $B$ à $A$ à l'aube (simultanément) pour se retrouver (sur la même route). Elles se sont rencontrées à midi, mais ne se sont pas arrêtées, et chacune a continué à marcher à la même vitesse. La première est arrivée (à $B$) à 16 heures, et la seconde (à $A$) à 21 heures. À quelle heure était l'aube ce jour-là ?
\end{problem}

\begin{problem}{6.}
L'hypoténuse d'un triangle rectangle (à l'examen standard américain) est de 10 pouces, et sa hauteur est de 6 pouces. Trouvez l'aire du triangle.

Les écoliers américains ont réussi à résoudre ce problème pendant dix ans, mais des écoliers russes sont arrivés de Moscou, et aucun d'entre eux n'a pu résoudre ce problème aussi bien que les écoliers américains (qui ont donné la réponse 30 pouces carrés). Pourquoi ?
\end{problem}

\begin{problem}{7.}
Vassia a deux sœurs de plus que de frères. Combien les parents de Vassia ont de filles de plus que de fils ?
\end{problem}

\begin{problem}{8.}
En Amérique du Sud, il existe un lac rond. Le 1er juin de chaque année, une fleur de Victoria Regia apparaît au centre du lac (la tige s'élève du fond et les pétales reposent sur l'eau, comme un nénuphar). Chaque jour, la surface de la fleur double et, le 1er juillet, elle recouvre enfin tout le lac : les pétales tombent et les graines coulent au fond. À quelle date la surface de la fleur est-elle la moitié de celle du lac ?
\end{problem}

\begin{problem}{9.}
Un paysan doit transporter un loup, une chèvre et un chou de l'autre côté d'une rivière en bateau, mais le bateau est si petit qu'il ne peut emporter qu'un seul des trois chargements. Comment les trois chargements peuvent-ils être transportés de l'autre côté de la rivière ? (Le loup ne peut pas être laissé seul avec la chèvre, et la chèvre ne peut pas être laissée seule avec le chou.)
\end{problem}

\begin{problem}{10.}
Un escargot grimpe sur un poteau de \SI{3}{\cm} pendant la journée et, la nuit, en s'endormant, il descend accidentellement de \SI{2}{\cm}. Le poteau mesure \SI{10}{\metre} de haut et, à son sommet, se trouve un bonbon savoureux pour l'escargot. Combien de jours faudra-t-il à l'escargot pour l'obtenir ?
\end{problem}

\begin{problem}{11.}
Un chasseur a marché \SI{10}{\km} au sud de sa tente, a tourné vers l'est, a marché tout droit vers l'est pendant encore \SI{10}{\km}, a tué un ours, a tourné vers le nord et, après avoir marché \SI{10}{\km} supplémentaires, s'est retrouvé à la tente. De quelle couleur était l'ours et où était tout cela ?
\end{problem}

\begin{problem}{12.}
Aujourd'hui à midi, la marée était haute. Quand sera-t-elle là demain ?
\end{problem}

\begin{problem}{13.}
Il y a deux volumes de Pouchkine sur l'étagère l'un à côté de l'autre : premier et deuxième. Les pages de chaque volume ont une épaisseur totale de \SI{2}{\cm}, et les couvertures mesurent \SI{2}{\cm} chacune. Un ver a rongé (perpendiculairement aux pages) de la première page du premier volume à Dernière page du deuxième volume. Quelle est la longueur du chemin que le ver a rongé ? [Ce problème de topologie a une réponse incroyable –– Le \SI{4}{\mm} est totalement inaccessible aux universitaires, mais certains Les enfants d’âge préscolaire peuvent le gérer facilement.]
\end{problem}

\begin{problem}{14.}
Trouvez un corps qui a cette apparence
Vue de dessus et de face (polyèdres). Dessinez une vue
de côté (les bords invisibles du polyèdre sont représentés
par une ligne pointillée).

\begin{figure}
	\footnotesize
	\null\hfill
	\parbox{0.2\linewidth}{\centering\includegraphics{resources/taskbook-99}\\Top view}
	\hfill
	\parbox{0.2\linewidth}{\centering\includegraphics{resources/taskbook-98}\\Front view}
	\hfill\null
\end{figure}
\end{problem}

\begin{problem}{15.}
De combien de façons le nombre 64 peut-il être représenté en une somme de 10 nombres entiers positifs si ils sont tous inférieurs ou égaux à 12 ?
[Les partitions qui ne diffèrent que par l'ordre des termes ne sont pas prises en compte.]
\end{problem}

\begin{problem}{16.}
En empilant une sur l'autre des pièces identiques (par exemple, des dominos), il est possible de former un auvent de longueur $x$. Quelle est la plus grande longueur possible $x$ de l'auvent ?
\begin{figure}
	\includegraphics{resources/taskbook-97}
\end{figure}
\end{problem}

\begin{problem}{17.}
La distance entre la ville $A$ et la ville $B$ est \SI{40}{\km}. Deux cyclistes ont quitté respectivement $A$ et $B$ au même moment,l'un vers l'autre, l'un à \SI{10}{\km\per\hour} et l'autre à \SI{15}{\km\per\hour}. Une mouche s'est envolée avec le premier cycliste à partir de $A$ à une vitesse de \SI{100}{\km\per\hour}, a volé vers le deuxième, a atterri sur son front et est revenu vers le premier, atterri sur son front, est revenu vers le second et ainsi de suite jusqu'à ce qu'ils entrent en collision, se frappent le front et écrasent la mouche. Combien de kilomètres la mouche a-t-elle parcouru au total ?
\begin{figure}
	\includegraphics{resources/taskbook-1}
\end{figure}
\end{problem}

\begin{problem}{18.}
One domino piece covers two squares of a chessboard.
Cover all the squares
except for two opposite ones (on the same diagonal) with 31 pieces. [A chessboard consists of $8 \times 8 = 64$ squares.]
\begin{figure}
	\includegraphics{resources/taskbook-2}
\end{figure}
\end{problem}

\begin{problem}{19.}
A caterpillar wants to slither from a corner of a cubic room (the left on the floor) to the opposite one
(the right on the ceiling).
Find the shortest route for such a journey along the walls of the room.
\begin{figure}
	\includegraphics{resources/taskbook-3}
\end{figure}
\end{problem}

\begin{problem}{20.}
You have two vessels of volumes 5~litres and 3~litres. Measure out one litre (obtaining it in one of the vessels).
\begin{figure}
	\includegraphics{resources/taskbook-4}
\end{figure}
\end{problem}

\begin{problem}{21.}
There are five heads and fourteen legs in a family. How many people and how many dogs are in the family?
\end{problem}

\begin{problem}{22.}
Equilateral triangles are constructed externally on sides $AB$, $BC$ and $CA$ of a triangle $ABC$.
Prove that their centres ($*$) form an equilateral triangle.
\begin{figure}
	\includegraphics{resources/taskbook-6}
\end{figure}
\end{problem}

\begin{problem}{23.}
What polygons may be obtained as sections of a cube by a plane? Can we get a pentagon? A heptagon?
A regular hexagon?
\begin{figure}
	\includegraphics{resources/taskbook-7}
\end{figure}
\end{problem}

\begin{problem}{24.}
Draw a straight line through the centre of a cube so that the sum of squares of the distances to it
from the eight vertices of the cube would be
a) maximal,
b) minimal (comparing with other such lines).
\end{problem}

\begin{problem}{25.}
A right circular cone is cut by a plane along a closed curve. Two balls inscribed into the cone
are tangent to the plane, one at point $A$ and the other at point $B$. Find a point $C$ on the cut line so
that the sum of the distances $CA + CB$ would be a) maximal, b) minimal.
\begin{figure}
	\includegraphics{resources/taskbook-9}
\end{figure}
\end{problem}

\begin{problem}{26.}
The Earth's surface is projected onto the cylinder formed by the lines tangent to the meridians
at their equatorial points along the rays parallel to the equator and passing through the Earth's pole axis.
Will the area of the projection of France be greater or smaller than the area of France itself?
\begin{figure}
	\includegraphics{resources/taskbook-10}
\end{figure}
\end{problem}

\begin{problem}{27.}
Prove that the remainder of division of the number $2^{p-1}$ by an odd prime $p$ is $1$.
(Examples: $2^2 = 3a + 1$, $2^4 = 5b+1$, $2^6 = 7c+1$, $2^{10} - 1 = 1023 = 11\cdot 93$.)
\end{problem}

\begin{problem}{28.}
A needle \SI{10}{\cm} long is thrown randomly onto lined paper with the distance between neighbouring
lines also \SI{10}{\cm}. This is repeated
$N$ (a million) times.
How many times (approximately, up to a few per
cent error) will the fallen needle intersect a line on the paper?
\begin{figure}
	\includegraphics{resources/taskbook-12}
\end{figure}
One can perform (as I did at the age of 10) this experiment with $N=100$ instead of a million throws.
	[The answer to this problem is surprising: $\frac2{\pi}N$. Moreover even for a bent needle of length $a \cdot \SI{10}{\cm}$ the number of intersections observed over $N$ throws will be approximately $\frac{2a}{\pi}N$.
		The number $\pi \approx  \frac{355}{113} \approx \frac{22}7.$]
\end{problem}

\begin{problem}{29.}
Polyhedra with triangular faces are, for example, Platonic solids: tetrahedron (4 faces),
octahedron (8 of them), icosahedron (20 -- and all the faces are the same; it is interesting to draw it,
it has 12 vertices and 30 edges).
\begin{figure}
	\footnotesize
	\null\hfill
	\parbox{0.3\linewidth}{\centering\includegraphics{resources/taskbook-131}\\Tetrahedron ($\text{tetra}= 4$)}
	\hfill
	\parbox{0.3\linewidth}{\centering\includegraphics{resources/taskbook-132}\\Octahedron ($\text{octo}= 8$)}
	\hfill\null\\
	{\Huge ?}\\Icosahedron
\end{figure}
Is it true that for any such (bounded convex polyhedron with triangular faces) the number of faces is
equal to twice the number of vertices minus four?


Yet another Platonic solid (there are 5 of them altogether):
\begin{figure}
	\includegraphics{resources/taskbook-14}
\end{figure}
\end{problem}

\begin{problem}{30.}
A dodecahedron is a convex polyhedron with twelve (regular) pentagonal
faces, twenty vertices
and thirty edges (its vertices are the centres of the faces of an icosahedron).
Inscribe into a dodecahedron five cubes (the vertices of each cube are vertices of the dodecahedron)
whose edges are diagonals of faces of the dodecahedron (a cube has 12~edges, one per face).
	[This was invented by Kepler for the sake of planets.]
\end{problem}

\begin{problem}{31.}
Find the intersection of two tetrahedra inscribed into a cube (so that the vertices of each are
vertices of the cube, and the edges are diagonals of the faces).
What fraction of the cube's volume is contained within the tetrahedra's intersection?
\end{problem}

\begin{problem}{31\textsuperscript{bis}.}
Construct the section of a cube by the plane passing through three given points on the edges.
	[Draw the polygon along which the planar section intersects the faces of the cube.]
\begin{figure}
	\includegraphics{resources/taskbook-15}
\end{figure}
\end{problem}

\begin{problem}{32.}
How many symmetries does a tetrahedron have? How many has a cube? octahedron? icosahedron?
dodecahedron? A symmetry is a transformation preserving lengths.
How many rotations are among them and how many reflections (in each of the five cases listed)?
\end{problem}

\begin{problem}{33.}
How many ways are there to paint $6$ faces of similar cubes with six colours $(1,\dotsc,6)$ [one per face]
so that no two of the coloured cubes obtained would be the same (could not be transformed one into another
by a rotation)?
\begin{figure}
	\includegraphics{resources/taskbook-17}
\end{figure}
\end{problem}

\begin{problem}{34.}
How many different ways are there to permute $n$ objects?
There are six of them for $n=3$: $(1,2,3)$, $(1,3,2)$, $(2,1,3)$, $(2,3,1)$, $(3,1,2)$, $(3,2,1)$. What if the number of objectis is $n=4$? $n=5$? $n=6$? $n=10$?
\begin{figure}
	\includegraphics{resources/taskbook-18}
\end{figure}
\end{problem}

\begin{problem}{35.}
A cube has $4$ long diagonals. How many different permutations of these four objects are obtained by rotations of a cube?
\begin{figure}
	\includegraphics{resources/taskbook-19}
\end{figure}
\end{problem}

\begin{problem}{36.}
The sum of the cubes of three integers is subtracted from the cube of the sum of these numbers. Is the difference always divisible by $3$?
\end{problem}

\begin{problem}{37.}
Same question for the fifth powers and divisibility by $5$, and for the seventh powers and divisibility by $7$.
\end{problem}

\begin{problem}{38.}
Calculate the sum
\begin{equation*}
	\frac{1}{1\cdot 2} + \frac{1}{2\cdot 3} + \frac{1}{3\cdot 4} + \dotsb + \frac{1}{99\cdot 100}
\end{equation*}
(with an error of not more than $1\%$ of the answer).
\end{problem}

\begin{problem}{39.}
If two polygons have equal areas, then they may be cut into a finite number of polygonal parts which may then be rearranged to obtain both the first and second polygons. Prove this! [For spatial solids this is not the case: the cube and tetrahedron of equal volumes cannot be cut this way!]
\begin{figure}
	\includegraphics{resources/q39_horizontal}
\end{figure}
\end{problem}

\begin{problem}{40.}
Four vertices of a parallelogram have been chosen at nodes of a piece of squared paper. It turns out that neither the parallelogram's sides nor its interior contain any other nodes of the squared paper. Prove that the area of such a parallelogram is equal to the area of one square of the paper.
\begin{figure}
	\includegraphics{resources/taskbook-24}
\end{figure}
\end{problem}

\begin{problem}{41.}
Under the conditions of question 40, $a$ nodes have turned out to be in the interior and $b$ on the sides of the parallelogram. Find its area.
\end{problem}

\begin{problem}{42.}
Is the statement analogous to question 40 true for parallelepipeds in 3-space?
\end{problem}

\begin{problem}{43.}
The rabbit (or Fibonacci) numbers form a sequence $1,2,3,5,8,\allowbreak 13,21,34,\dotsc$, in which $a_{n+2}=a_{n+1}+a_n$ for any
$n=1,2,\dotsc$ ($a_n$ is the $n$-th number in the sequence). Find the greatest common divisor of the numbers $a_{100}$ and $a_{99}$.
\end{problem}

\begin{problem}{44.}
Find the (Catalan) number of ways to cut a convex $n$-gon into triangles by cutting along its non-intersecting diagonals.
For example, $c(4)=2$, $c(5)=5$, $c(6)=14$. How can one find $c(10)$?
\begin{figure}
	\includegraphics{resources/taskbook-281}
	\qquad
	\includegraphics{resources/taskbook-282}
\end{figure}
\end{problem}

\begin{problem}{45.}
A cup tournament has $n$ participating teams, each losing team leaves, and the overall winner is decided after $n-1$ games.
The tournament schedule may be written symbolically as, for instance,  $((a,(b,c)),d)$ meaning $b$ plays $c$, the winner meets $a$, and the winner of those meets $d$].
What is the number of different schedules for 10 teams?
\begin{itemize}
	\item For 2 teams, we have only $(a,b)$, and the number is 1.
	\item For 3 teams, there are only $((a,b),c)$, or $((a,c),b)$, or $((b,c),a)$, and the number is 3.
	\item For 4 teams:
	      \begin{equation*}
		      \begin{array}{@{}cccc@{}}
			      (((a,b),c),d) & \quad\;(((a,c),b),d) & \quad\;(((a,d),b),c) & \quad\;(((b,c),a),d) \\
			      (((b,d),a),c) & \quad\;(((c,d),a),b) & \quad\;(((a,b),d),c) & \quad\;(((a,c),d),b) \\
			      (((a,d),c),b) & \quad\;(((b,c),d),a) & \quad\;(((b,d),c),a) & \quad\;(((c,d),b),a) \\
			      ((a,b),(c,d)) & \quad\;((a,c),(b,d)) & \quad\;((a,d),(b,c))
		      \end{array}
	      \end{equation*}
\end{itemize}
\end{problem}

\begin{problem}{46.}
Join $n$ points $1, 2, \dotsc, n$ by intervals ($n-1$ of them) to obtain a tree. How many different trees may be obtained (the $n=5$ case is already interesting!)?

$n=2$:\quad \includegraphics{resources/taskbook-291}\,,\quad the number is 1;

$n=3$:\quad
\includegraphics{resources/taskbook-292}\,,\quad
\includegraphics{resources/taskbook-293}\,,\quad
\includegraphics{resources/taskbook-294}\,,\quad
the number is 3;

$n=4$:\quad\def\quad{\hskip.7em}
$\vcenter{\hbox{\includegraphics{resources/taskbook-295}}}$,\quad
$\vcenter{\hbox{\includegraphics{resources/taskbook-296}}}$,\quad
$\vcenter{\hbox{\includegraphics{resources/taskbook-297}}}$,\quad
$\vcenter{\hbox{\includegraphics{resources/taskbook-298}}}$,\quad
$\vcenter{\hbox{\includegraphics{resources/taskbook-299}}\hbox{\includegraphics{resources/taskbook-290}}
		\vskip-8pt
		\hbox to50bp{\dotfill}}$,\\
\null\hspace{\parindent}\phantom{$n=4$:}\quad the number is 16.
\end{problem}

\begin{problem}{47.}
A permutation $(x_1,x_2, \dotsc,x_n)$ of numbers $\{1, 2, \dotsc, n\}$ is called a
\emph{snake} (of length $n$) if $x_1<x_2>x_3<x_4 \dotsb$.

\begin{note}{Example:}
	\begin{equation*}
		\begin{aligned}[t]
			 & \begin{aligned}[t] n=2, \text{\ \ only\ \ } 1<2, \end{aligned}                                                &                         & \text{the number is }1,   \\
			 & \hskip-\nulldelimiterspace\mathord{\left.\begin{aligned} n=3, \hphantom{\text{\ \ only\ \ }} 1 & <3>2 \\
                2                                     & <3>1\end{aligned} \right\}},   &                         & \text{the number is }2, \\
			 & \hskip-\nulldelimiterspace\mathord{\left.\begin{aligned} n=4, \hphantom{\text{\ \ only\ \ }} 1 & <3>2<4 \\
                1                                     & <4>2<3 \\
                2                                     & <3>1<4 \\
                2                                     & <4>1<3 \\
                3                                     & <4>1<2\end{aligned} \right\}},
			 &                                                                                                               & \text{the number is }5.                             \\
		\end{aligned}
	\end{equation*}
\end{note}
Find the number of snakes of length $10$.
\end{problem}

\begin{problem}{48.}
Let $s_n$ be the number of snakes of length $n$:
\begin{equation*}
	s_1=1, \quad s_2=1, \quad s_3=2, \quad s_4=5, \quad s_5=16, \quad s_6=61.
\end{equation*}
Prove that the Taylor series of the tangent is
\begin{equation*}
	\tan x=1\, \frac{x^1}{1!}+2\, \frac{x^3}{3!}+16\, \frac{x^5}{5!}+\dots=
	\textstyle\sum\limits_{k=1}^{\infty} s_{2k-1}\, \frac{x^{2k-1}}{(2k-1)!}.
\end{equation*}
\end{problem}

\begin{problem}{49.}
Find the sum of the series
\begin{equation*}
	1+1\, \frac{x^2}{2!}+5\, \frac{x^4}{4!}+61\, \frac{x^6}{6!}+\dots=
	\textstyle\sum\limits_{k=0}^{\infty} s_{2k}\,\frac{x^{2k}}{(2k)!}.
\end{equation*}
\end{problem}

\begin{problem}{50.}
For $s>1$, prove the identity:
\begin{equation*}
	\textstyle\prod\limits_{p=2}^{\infty} \frac{1}{1-\frac{1}{p^s}}=\textstyle\sum\limits_{n=1}^{\infty} \frac{1}{n^s}.
\end{equation*}
(The product is over all prime numbers $p$, and the summation over all natural numbers~$n$.)
\end{problem}

\begin{problem}{51.}
Find the sum of the series:
\begin{equation*}
	1+ \frac{1}{4}+ \frac{1}{9}+\dots=\textstyle\sum\limits_{n=1}^{\infty} \frac{1}{n^2}.
\end{equation*}
[Prove that it is $\nicefrac{\pi^2}{6}$, that is, approximately $\nicefrac{3}{2}$.]
\end{problem}

\begin{problem}{52.}
Find the probability of the irreducibility of a fraction $\nicefrac{p}{q}$ (this is defined as follows:
in the disk $p^2+q^2 \leqslant R^2$, we count the number $N$ of vectors with integer
$p$ and $q$ not having a common divisor greater than 1, after which the probability of the irreducibility is the
limit of the ratio $\nicefrac{N(R)}{M(R)}$, where $M(R)$ is the number of integer points in the disk $(M \sim \pi R^2)$).
\begin{figure}
	\includegraphics{resources/taskbook-36}\\
	\footnotesize $M(5)=81$, $N(5)=44$, $\nicefrac{N}{M} = \nicefrac{44}{81}$
\end{figure}
\end{problem}

\begin{problem}{53.}
For the sequence of Fibonacci numbers $a_n$ from problem 43, find the limit of the ratio
$a_{n+1}/a_n$ when $n$ tends to infinity:\vspace{2\jot}
\begin{equation*}
	\frac{a_{n+1}}{a_n}=2,\ \frac 32,\ \frac53, \ \frac85, \ \frac{13}8,
	\ \frac{34}{21}.
\end{equation*}
[The answer is \enquote{the golden ratio},
$\frac{\sqrt{5}+1}{2\vphantom)} \approx 1.618$. This is the ratio of the sides of a card which stays
similar to itself after cutting off the square whose side is the smaller side of the card,
$\frac{AB}{BC}=\frac{PC}{CD}$. How is the golden ratio related to a regular pentagon and a five-pointed star?]
\begin{figure}
	\includegraphics{resources/taskbook-37}
\end{figure}
\end{problem}

\begin{problem}{54.}
Calculate the infinite continued fraction
\begin{equation*}
	1+\cfrac{1}{2+\cfrac{1}{1+\cfrac{1}{2+\cfrac{1}{1+\cfrac{1}{2+\ldots}}}}}=
	a_0+\cfrac{1}{a_1+\cfrac{1}{a_2+\cfrac{1}{a_3+\dots}}}
\end{equation*}
with $a_{2k}=1$ and $a_{2k+1}=2$ (that is, find the limit of the fractions
\begin{equation*}
	a_0+\cfrac{1}{a_1+\cfrac{1}{a_2+{\atop{\ddots \atop {}} + \cfrac{1}{a_n}}}}
\end{equation*}
for $n \to \infty$).
\end{problem}

\begin{problem}{55.}
Find the polynomials
\begin{equation*}
	y=\cos 3 (\arccos x),\ y=\cos 4 (\arccos x),\
	y=\cos n (\arccos x),
\end{equation*}
where $|x| \leqslant 1$.
\end{problem}

\begin{problem}{56.}
Calculate the sum of the $k$-th powers of the $n$ complex  $n$-th roots of unity.
\end{problem}

\begin{problem}{57.}
On the $(x,y)$-plane, draw the curves defined parametrically by:
\begin{equation*}
	\{x=\cos 2t, y=\sin 3t\},\quad
	\{x=t^3-3t, y=t^4-2t^2\}.
\end{equation*}
\vspace{-2\baselineskip}%remove this vertical space if your translation has text coming after the equation
\end{problem}

\begin{problem}{58.}
Calculate $\int_0^{2\pi} \sin^{100} x\,dx$ (with an error of not more than 10\% of the answer).
\end{problem}

\begin{problem}{59.}
Calculate $\int_1^{10} x^x\,dx$ (with an error of not more than 10\% of the answer).
\end{problem}

\begin{problem}{60.}
Find the area of a triangle with angles $(\alpha, \beta, \gamma)$ on a radius 1 sphere,
whose sides are great circles (sections of a sphere by planes passing through its centre).

\begin{note}{Answer:}
	$S=\alpha+\beta+\gamma-\pi$ (for example, for a triangle with
	three right angles, $S=\nicefrac{\pi}{2}$, that is, one-eighth of the total area of the sphere).
	\begin{figure}
		\null\hfill
		\includegraphics{resources/taskbook-44}
		\hfill
		\includegraphics{resources/taskbook-45}
		\hfill\null
	\end{figure}
\end{note}
\end{problem}

\begin{problem}{61.}
A circle of radius $r$ rolls (without slipping) inside a circle of radius 1.
Draw the whole trajectory of a point of the rolling circle (this trajectory is called a hypocycloid)
for $r=\nicefrac{1}{3}$, for $r=\nicefrac{1}{4}$, for $r=\nicefrac{1}{n}$, and for $r=\nicefrac{1}{2}$.
\end{problem}

\begin{problem}{62.}
In a class of $n$ pupils, estimate the probability of there being two pupils with the same birthdays. Is it high or is it low?

\begin{note}{Answer:}
	(very) high if the number of the pupils is (well) above $n_0$,
	(very) low if it is (well) below $n_0$, and what this $n_0$ actually is
	(when the probability $p \approx \nicefrac{1}{2}$) is to be found.
\end{note}
\end{problem}

\begin{problem}{63.}
Snell's (Snellius') law states that the angle $\alpha$ made by a ray of light with the normal to layers of a stratified medium satisfies the equation
\begin{equation*}
	n(y) \sin \alpha=\text{const},
\end{equation*}
where $n(y)$ is the refractive index of the layer at height $y$ (the quantity $n$ is	inversely proportional to the speed
of light in the medium when taking its speed in vacuum as 1; in water $n=nicefrac{4}{3}$).
\begin{figure}
	\null\hfill
	\includegraphics{resources/taskbook-47}
	\hfill
	\includegraphics{resources/taskbook-471}
	\hfill\null
\end{figure}

Draw ray trajectories in the medium  \enquote{air over a desert}, where the index $n(y)$ has a maximum
at a certain height.
(A solution to this problem explains mirages in a desert to those understanding how trajectories of rays emanating from objects are related to the images.)
\end{problem}

\begin{problem}{64.}
Inscribe into an acute-angled triangle $ABC$ a triangle $KLM$ of minimal perimeter
(with its vertex $K$ on $AB$, $L$ on $BC$, $M$ on $CA$).
\begin{figure}
	\includegraphics{resources/taskbook-48}
\end{figure}

\begin{note}{Hint:}
	The answer for non-acute-angled triangles is not similar to the beautiful answer for acute-angled ones.
\end{note}
\end{problem}

\begin{problem}{65.}
Calculate the mean value of the function  $\nicefrac{1}{r}$ (where
$r^2=x^2+y^2+z^2$, $r$ is the distance to the origin) on the radius
$R$ sphere centred at the point $(X,Y,Z)$.

\begin{note}{Hint:}
	The problem is related to Newton's gravitation law and Coulomb's law of electricity theory.
	In the two-dimensional version of the problem, the function should be replaced by $\ln r$, and the sphere by the circle.
\end{note}
\end{problem}

\begin{problem}{66.}
The fact $2^{10}=1024 \approx 10^3$ implies
$\log_{10} 2 \approx 0.3$. 	Estimate how much they differ, and calculate $\log_{10} 2$ to three decimal places.
\end{problem}

\begin{problem}{67.}
Find $\log_{10} 4$, $\log_{10} 8$,
$\log_{10} 5$, $\log_{10} 50$, $\log_{10} 32$, $\log_{10} 128$,
$\log_{10} 125$, $\log_{10} 64$ with the same precision.
\end{problem}

\begin{problem}{68.}
Using $7^2 \approx 50$, find an approximate value of $\log_{10} 7$.
\end{problem}

\begin{problem}{69.}
Knowing $\log_{10} 64$ and $\log_{10} 7$, find $\log_{10} 9$, $\log_{10} 3$,
$\log_{10} 27$, $\log_{10} 6$, $\log_{10} 12$.
\end{problem}

\begin{problem}{70.}
Using $\ln (1+x) \approx x$ ($\ln$ is $\log_e$), find $\log_{10} e$ and
$\ln 10$ from the relation\footnote{The Euler number $e = 2{.}71828\dots$ is defined as the limit of the sequence
	$\left(1+\frac{1}{n}\right)^n$ for $n\to \infty$, and is equal to the sum of the series
	$1+\frac{1}{1!} +\frac{1}{2!}+\frac{1}{3!}+\dotsb$. It may also be defined via the quoted formula for
	$\ln (1+x)$: $\lim\limits_{x\to 0}\frac{\ln(1+x)}{x} = 1$.}
%
\begin{equation*}
	\log_{10} a=\frac{\ln a}{\ln 10}
\end{equation*}
and from the values of $\log_{10} a$ calculated earlier (for example, for $a=128/125, 1024/1000$
and so on).

	[Solutions to problems 65--69 deliver after half an hour a table of four-digit logarithms of any numbers using
		products of numbers found already as basic data and the formula
		\begin{equation*}
			\ln (1+x) \approx x-\frac{x^2}{2}+\frac{x^3}{3}-\frac{x^4}{4}+\dotsb,
		\end{equation*}
		for corrections.] (In this way Newton compiled a table of
40-digit logarithms!)
\end{problem}

\begin{problem}{71.}
Consider the sequence of powers of two: $1$, $2$, $4$, $8$, $16$, $32$, $64$,
$128$, $256$, $512$, $1024$, $2048, \dotsc$ Among the first twelve numbers, four have their decimal expression
starting with 1, and none has it starting with 7.

Prove that in the limit $n \to \infty$ the first digit of the numbers $2^m$,
$0\leqslant m \leqslant n$, will be met with a certain frequency:
$p_1 \approx 30\%, p_2 \approx 18\%, \dotsc, p_9 \approx 4\%$.
\end{problem}

\begin{problem}{72.}
Verify the behaviour of the first digits of powers of three: $1,
	3, 9, 2, 8, 2, 7, \dotsc$ Prove that, in the limit here, we also
get certain frequencies and, moreover, the same as for the powers of two.
Find an exact formula for $p_1, \dotsc, p_9$.

\begin{note}{Hint:}
	The first digit of a number $x$ is determined by the fractional part
	of the number $\log_{10} x$, therefore one has to consider the sequence of the fractional parts of
	the numbers $m \alpha$, where $\alpha=\log_{10} 2$.
\end{note}
Prove that these fractional parts are distributed over the interval from 0 to~1
uniformly: out of the $n$ fractional parts of the numbers $m \alpha$, $0 \leqslant m<n$,
a subinterval $A$ will contain the quantity~$k_n (A)$ such that, for $n \to \infty$,
$\lim (k_n (A)/n)=(\text{the length of the subinterval~$A$})$.
\end{problem}

\begin{problem}{73.}
Let $g\colon M \to M$ be a smooth map of a bounded domain $M$ onto itself which
is one-to-one and preserves areas (volumes in the multi-dimensional case) of domains.

Prove that in any neighbourhood $U$ of any point of $M$ and for any $N$ there exists a point $x$
such that $g^T x$ is also in $U$ for a certain integer $T>N$ (\enquote{the recurrence theorem}).
\end{problem}

\begin{problem}{74.}
Let $M$ be the torus surface (with coordinates $\alpha \pmod{2\pi}$, $\beta \pmod{2\pi}$),
and
\begin{equation*}
	g(\alpha, \beta)=(\alpha+1, \beta+ \sqrt{2}) \pmod{2\pi}.
\end{equation*}
Prove that the sequence of points
$\{g^T (x)\}$, $T=1, 2, \dotsc$, is everywhere dense in the torus.
\end{problem}

\begin{problem}{75.}
In the notation of problem 74, let
\begin{equation*}
	f(\alpha, \beta)=(2\alpha+\beta,\alpha+\beta) \pmod{2\pi}.
\end{equation*}
Prove that there is an everywhere dense subset of the torus consisting of periodic points $x$ (that is, such that
$f^{T (x)} x=x$ for certain integer $T>0$).
\end{problem}

\begin{problem}{76.}
In the notation of problem 74 prove that, for almost all points $x$ of the torus,
the sequence of points $\{g^T (x)\}$, $T=1, 2, \dotsc$, is everywhere dense in the torus
(points $x$ without this property constitute a set of measure zero).
\end{problem}

\begin{problem}{77.}
In problems 74 and 76 prove that the sequence $\{g^T (x)\}$, $T=1, 2, \dotsc$, is distributed
over the torus uniformly: if a domain $A$ contains $k_n(A)$ points out of the $n$ with $T=1, 2, \dotsc,n$, then
\begin{equation*}
	\lim_{n \to \infty} \frac{k_n(A)}{n}=\frac{\operatorname{mes} A}{\operatorname{mes} M}
\end{equation*}
(for example, for a Jordan measurable domain $A$ of measure $\operatorname{mes} A$).
\end{problem}

\vfill
\begin{note}{Note to problem 13.}
	I tried to illustrate with this problem the difference in approaches to tasks by mathematicians and physicists, in my invited paper in the journal \enquote{Physics -- Uspekhi} for the 2000 Christmas jubilee. My success surpassed by far what I intended: the editors, unlike preschoolers, on whom I had been basing my experience, failed to solve the problem, and therefore altered it to match my \SI{4}{\mm} answer as follows: instead of \enquote{from the first page of volume 1 to the last page of volume 2} they typeset \enquote{from the \emph{last} page of volume 1 to the \emph{first} page of volume 2}.

	This true story is so implausible that I am including it here: the proof is the editors' version published by the journal.
\end{note}
\clearpage
\null\vfill
\noindent
Translation Russian - English:\\
\null\quad Victor Goryunov and Sabir Gusein-Zade\\
\\
Design and layout:\\
\null\quad Konrad Renner and Christian Stussak\\
\\
\\
From the Russian edition:\\
\null\quad \textrussian{В. И. Арнольд: Задачи для детей от 5 до 15 лет}\\
\null\quad Moscow, MCCME, 2004\\
\null\quad ISBN 5-94057-183-2\\
\\
\\
Picture credits title page:\\
\null\quad Archives of the Mathematischen Forschungsinstituts Oberwolfach\\
\\
Version:\\
\null\quad \today\\
\\
This book is available under the CC BY-NC-SA 3.0 license at the IMAG\-I\-NARY platform: \href{http://www.imaginary.org/background-materials}{www.imaginary.org/background-materials}.\\
IMAGINARY is a project by the Mathematisches Forschungsinstitut Oberwolfach supported by the Klaus Tschira Stiftung.
\end{document}
